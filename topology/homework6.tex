\documentclass{article}
\usepackage{graphicx}
\usepackage{amsfonts}

\begin{document}

\title{Homework 6}
\author{Josh Cai}

\maketitle


\begin{enumerate}
\item
We see that the set of metric open balls $B_x^d(\epsilon)$ for $0<\epsilon<1$ forms a basis for the topology on $X$. By definition, the set of metric open balls $B_x^{\bar d}(\epsilon)$ for $0<\epsilon<1$ also forms a basis for the topology on $X$. Since these are the same collection of metric open balls, we see that the topology induced by both metrics is the same.

\item
(1) is finer than (3) and (5)
\\(2) is finer than (1), (3), and (5)
\\(4) is finer than (2),(1),(3),(5)
\\\\We show that (4) is finer than (2), (2) is finer than (1), (1) is finer than (5), and (1) is finer than (3):
\\\\Given a basis element $(a,b)-K$ of $\mathcal{T}_2$ and a point $x$ of $(a,b)-K$, we take the basis element $(c,x]$, where $c<x$ is an irrational number greater than the largest rational number smaller than $x$. $(c,x]$ contains $x$ and lies within $(a,b)-K$. However, no basis element in $\mathcal{T}_2$ lies within $(c,x]$ and contains $x$. This shows $\mathcal{T}_4$ is strictly finer than $\mathcal{T}_2$.
\\\\Given a basis element $(a,b)$ of $\mathcal{T}_1$ and a point $x$ of $(a,b)$, we take the basis element $(a,b)$ of $\mathcal{T}_2$ since it contains $x$ and lies within $(a,b)$. This shows $\mathcal{T}_2$ is finer than $\mathcal{T}_1$.
\\\\Given a basis element $(-\infty,a)$ of $\mathcal{T}_5$ and a point $x$ of $(-\infty,a)$, we take the basis element $(-\infty,a)$ of $\mathcal{T}_1$ since it contains $x$ and lies within $(-\infty,a)$. However, no basis element in $\mathcal{T}_5$ can lie within $(0,\infty) \in \mathcal{T}_1$, so $\mathcal{T}_1$ is strictly finer than $\mathcal{T}_5$.
\\\\Given a basis element that is infinite of $\mathcal{T}_3$ and a point $x$ in it, we take the same basis element in $\mathcal{T}_1$, since it lies in the first basis element and contains $x$. However, no basis element in $\mathcal{T}_3$ lies within $(a,b)$, since any interval within would have a complement that is infinite. This shows $\mathcal{T}_1$ is strictly finer than $\mathcal{T}_3$.

\item
a) $\emptyset \in \bigcap \{\mathcal{T}_{\alpha}\}_{\alpha \in I}$ because $\emptyset$ is in $\mathcal{T}_{\alpha}$ for all $\alpha \in I$ since each $\mathcal{T}_{\alpha}$ is a topology.
\\b) $X \in \bigcap \{\mathcal{T}_{\alpha}\}_{\alpha \in I}$ because $X$ is in $\mathcal{T}_{\alpha}$ for all $\alpha \in I$ since each $\mathcal{T}_{\alpha}$ is a topology.
\\c) Let $Y$ be a subcollection in $\bigcap \{\mathcal{T}_{\alpha}\}_{\alpha \in I}$. Then $\bigcup Y \in \bigcap \{\mathcal{T}_{\alpha}\}_{\alpha \in I}$ because if $Y$ is an element of $\mathcal{T}_{\alpha}$ for all ${\alpha \in I}$, then $\bigcup Y \in \mathcal{T}_{\alpha}$ for all ${\alpha \in I}$ since each $\mathcal{T}_{\alpha}$ is a topology. Thus, $\bigcup Y \in \bigcap \{\mathcal{T}_{\alpha}\}_{\alpha \in I}$.
\\d) Let $Y$ be a finite subcollection in $\bigcap \{\mathcal{T}_{\alpha}\}_{\alpha \in I}$. Then $\bigcap Y \in \bigcap \{\mathcal{T}_{\alpha}\}_{\alpha \in I}$ because if $Y$ is an element of $\mathcal{T}_{\alpha}$ for all ${\alpha \in I}$, then $\bigcap Y \in \mathcal{T}_{\alpha}$ for all ${\alpha \in I}$ since each $\mathcal{T}_{\alpha}$ is a topology. Thus, $\bigcap Y \in \bigcap \{\mathcal{T}_{\alpha}\}_{\alpha \in I}$.
\\\\Since the intersection of a family of topologies satisfies all the properties of a topology, it is a topology.

\item
Given a basis element $(a,b]$ of $\mathcal{C}$ and a point $x$ of $(a,b]$, then the basis element of the upper limit topology $(a,x]$ lies in $(a,b]$ and contains $x$. However, given the basis element $(a,x]$ in the upper limit topology, where $x$ is irrational, there is no basis in $\mathcal{C}$ that lies in $(a,x]$ and contains $x$. Therefore, $\mathcal{C}$ is strictly coarser than the upper limit topology.


\end{enumerate}

\end{document}