\documentclass{article}
\usepackage{graphicx}
\usepackage{amsfonts}

\begin{document}

\title{Homework 5}
\author{Josh Cai}

\maketitle


\begin{enumerate}
\item
(i) Positive definiteness: The absolute value guarantees that all values are non-negative. If $d((x_1,...x_n),(y_1,...,y_n))=0$, then for every $i<n$, $x_i = y_i$, which means $(x_1,...x_n)=(y_1,...,y_n)$. If $(x_1,...x_n)=(y_1,...,y_n)$, then by definition $d((x_1,...x_n),(y_1,...,y_n))=0$.
\\(ii) Symmetry: The absolute value preserves symmetry, since $|x_i - y_i|$ = $|y_i - x_i|$. 
\\(iii) Triangle inequality: For each $i$, it is true that $|x_i-z_i|\le |x_i-y_i|+|y_i-z_i|\le d(x,y)+d(y,z)$ where $x, y, z$ are ordered pairs. Since for every $i$ the statement holds, we find the maximum $|x_i-z_i| = d(x,z)$ and know that it must be less than $d(x,y)+d(y,z)$.

\item
The unit ball looks like a circle with radius $r$ under the standard metric. The unit ball looks like a square with vertices at (0,$r$), ($r$,0), (0,$-r$), ($-r$,0) under the taxicab metric. The unit ball looks like a square with vertices at ($r$,$r$), ($r$,$-r$), ($-r$,$r$), ($-r$,$-r$) under the greatest distance metric.

\item
The cofinite topology is $\mathcal{T}:=\{Y\subset\mathbb{R}|\mathbb{R}-Y$ is finite or $Y=\emptyset\}$. For every $U$ in $\mathcal{T}$, $U \subset \mathbb{R}$, so $U$ is in the standard topology on $R$. To show that the standard topology is \textit{strictly} finer, we find that $\{0,1\}$ is in the standard topology but not in $\mathcal{T}$. 

\item
Let $\mathcal{T}$ be the metric topology on $\mathbb{R}^2$ induced by the greatest component distance metric, $d_1$. Let $\mathcal{T'}$ be the metric topology on $\mathbb{R}^2$ induced by the taxicab metric, $d_2$. Let the two points $\vec{x},\vec{y}$ be $\vec{x}=(x_1,x_2)$ and $\vec{y}=(y_1,y_2)$. Let $\delta_i = |y_i - x_i|$. We know that \[d_1(\vec{x},\vec{y}) = \textrm{max}(\delta_1, \delta_2)\] and \[d_2(\vec{x},\vec{y}) = \delta_1+\delta_2.\]
It is easy to see that max$(\delta_1, \delta_2) \le \delta_1+\delta_2$ since $\delta_i$ is non-negative. Therefore, 
\[d_1(\vec{x},\vec{y}) \le d_2(\vec{x},\vec{y}).\] So $\mathcal{T} \subset \mathcal{T'}$.
\\\\Let $\delta_m = $ max($\delta_1,\delta_2$). Then 
\[d^2_2(\vec{x},\vec{y})=\delta^2_1+\delta^2_2+2\delta_1\delta_2 \le \delta^2_m+\delta^2_m+2\delta^2_m = 4\delta^2_m\]
So, $d_2(\vec{x},\vec{y}) \le 2d_1(\vec{x},\vec{y})$ which means $\mathcal{T} \supset \mathcal{T'}$. Thus $\mathcal{T} = \mathcal{T'}$. Since the topology induced by the taxicab metric is the same as the standard topology, we find that the topology induced by the greatest component distance metric is the same as the standard topology.



\end{enumerate}

\end{document}