\documentclass{article}
\usepackage{graphicx}
\usepackage{amsfonts}

\begin{document}

\title{Homework 7}
\author{Josh Cai}

\maketitle


\begin{enumerate}
\item
a) Let $x\in S(a,b)$. Then $s = an_1+b$ for some $n_1 \in \mathbb{Z}$. \[S(a,x) = \{an+an_1+b|n \in \mathbb{Z}\}\]
\[=\{a(n+n_1)+b|n\in\mathbb{Z}\}\]
Since $n+n_1=n_2 \in \mathbb{Z}$, $S(a,x) = \{an_2+b|n_2\in\mathbb{Z}\} = S(a,b)$.
\\\\b)Let $z \in S(a,x)\cap S(b,x)$. Then $z = an_1+x = bn_2+x$ for $n_1,n_2 \in \mathbb{Z}$. Then $z-x = an_1 = bn_2$. Since $a|z-x$ and $b|z-x$, clearly lcm$(a,b)|z-x$. So $z-x =$lcm$(a,b)n$ for some $n \in \mathbb{Z}$. Then $z =$lcm$(a,b)n+x$, so $z\in S($lcm$(a,b),x)$. Since $z$ was arbitrary, $S($lcm$(a,b),x) \supset S(a,x)\cap S(b,x)$.
\\\\Let $z\in S($lcm$(a,b),x)$. Then $z =$lcm$(a,b)n+x$ for some $n\in\mathbb{Z}$. Since $b|$lcm$(a,b)$, lcm$(a,b)$ = $bd$ for some $d\in\mathbb{Z}$. Then $z = bdn+x$, and since $dn = n_1 \in\mathbb{Z}$, $z = bn_1+x\in S(b,x)$. By a similar argument, $z \in S(a,x)$. Since $z$ was arbitrary, $S($lcm$(a,b),x) \subset S(a,x)\cap S(b,x)$.
\\Since they are subsets of each other, $ S($lcm$(a,b),x) = S(a,x)\cap S(b,x)$.
\\\\c) $S(1,0) = \{1*n+0|n\in\mathbb{Z}\} = \{n|n\in\mathbb{Z}$, which we clearly see is the the set $\mathbb{Z}$. Since $S(1,0)\in\mathcal{B}$, it is obvious that $\bigcup\mathcal{B} \subset \mathbb{Z}$. Let $x\in \mathbb{Z}$. Then $x \in \S(1,0)$ since $x = 1*x + 0$. Then, $\mathbb{Z}\subset\bigcup\mathcal{B}$, so $\mathbb{Z}\subset\bigcup\mathcal{B}$.
\\\\If $\mathcal{T}$ is the topology consisting of all arithmetic progressions, then suppose $z$ is one such arithmetric progression. For each $y \in z$, we find $S(0,y) \in \mathcal{B}$. Thus, a union of elements in $\mathcal{B}$ is equal to every open set in $\mathcal{T}$. Therefore, $\mathcal{B}$ is a basis for $\mathcal{T}$.

\item
a)I believe that there is a mistake in the problem: given the values $z_1 = -5$ and $z_2 = 3$, we find that the left hand side is $|-5+3| = |-2| = 1/2$ and the right hand side is max$(|-5|,|3|)=$max$(1/5,1/3)=1/3$. However, $1/2 > 1/3$, so the inequality cannot be true.
\\\\b) Positive definiteness: $d(x,y) = |x-y|$ which is 0 is $x=y$, or $1/$abs$(x-y)>0$ if not (abs is the absolute value function). 
\\\\Symmetry: $d(x,y) = |x-y| = 1/$abs$(x-y) = 1/$abs$((-1)(x-y)) = 1/$abs$(y-x)=|y-x| = d(y,x)$.
\\\\Triangle inequality: We need to show $d(x,y)+d(y,z) \ge d(x,z)$.
By definition, we have \[\frac{1}{abs(x-y)} + \frac{1}{abs(y-z} \ge \frac{1}{abs(x-z)}\]
\[\Leftrightarrow\frac{abs(x-y)+(y-z)}{abs(x-y)abs(y-z)} \ge \frac{1}{abs(x-z)}\]
\[\Leftrightarrow\frac{abs(x-z)(abs(x-y)+(y-z))}{abs(x-y)abs(y-z)} \ge 1\]
which is true by the AM-GM inequality. 
\item
b) Since $y > 0$, \[y+x > 0+x \;\textrm{(axiom 6)}\]
\[\Leftrightarrow y+x>x+0 \;\textrm{(axiom 2)}\]
\[\Leftrightarrow y+x>x+0>0+0 \;\textrm{(axiom 6 and}\;x>0)\]
\[\Leftrightarrow y+x>x+0>0 \;\textrm{(axiom 3)}\]
\[\Leftrightarrow x+y>x+0>0 \;\textrm{(axiom 2)}\]
\[\Leftrightarrow x+y>0\]
\\Since $x>0$ and $y>0$, then \[x\cdot y > 0 \cdot y\;\textrm{(axiom 6)}\]
\[\Leftrightarrow x\cdot y>0 \;\textrm{(\#1b)}\]
\\\\d)\[x>y\]
\[\Leftrightarrow x-x > y-x \; \textrm{(axiom 6)}\]
\[\Leftrightarrow 0 > y-x \; \textrm{(axiom 4)}\]
\[\Leftrightarrow 0-y > y-x-y \; \textrm{(axiom 6)}\]
\[\Leftrightarrow 0-y > -x+y-y \; \textrm{(axiom 2)}\]
\[\Leftrightarrow 0-y > -x \; \textrm{(axiom 4)}\]
\[\Leftrightarrow -y+0 > -x \; \textrm{(axiom 2)}\]
\[\Leftrightarrow -y > -x \; \textrm{(axiom 3)}\]
\\\\e)Since $z<0$, let $z'=(-1)(z)>0$ (from )
\[x>y\]
\[\Leftrightarrow -x<-y\; \textrm{(\#2d)}\]
\[\Leftrightarrow (-1)x<(-1)y\; \textrm{(\#1f)}\]
\[\Leftrightarrow (-1)xz'<(-1)yz'\; \textrm{(axiom 6) since }z'>0\]
\[\Leftrightarrow x(-1)z'<y(-1)z'\; \textrm{(axiom 2)}\]
\[\Leftrightarrow x(-z')<y(-z')\; \textrm{(\#1f)}\]
\[\Leftrightarrow xz<yz\; \textrm{(definition of }z')\]
\\\\f) By trichotomy, $x>0, x=0,$ or $x<0$. Since $x\neq 0$ by assumption, we check the other two cases. \\If $x>0$, $x^2 = x\cdot x > 0$ by \#2b (proved above). 
\\If $x<0$, then $0>x$ which means \[0\cdot x < x \cdot x\;\textrm{(\#2e)}\]
\[\Leftrightarrow 0<x\cdot x\;\textrm{(\#1b)}\]
\[\Leftrightarrow x^2>0.\]

\end{enumerate}

\end{document}