\begin{enumerate}

\item[\textbf{6.2}] 
\textit{(i)}
$\dfrac{1}{2z+5}=$
$\displaystyle\sum\limits_{n=0}^\infty (-1)^n\dfrac{2^n}{5^{n+1}}$, $|z| < 5/2$
\\\\
\textit{(ii)}
$\dfrac{1}{1+z^4}=$
$\displaystyle\sum\limits_{n=0}^\infty (-1)^nz^{4n}$, $|z| < 1$
\\\\
\textit{(iii)}
$\dfrac{1+\mathrm{i}z}{1-\mathrm{i}z}=$
$\dfrac{1}{1-\mathrm{i}z} - \dfrac{\mathrm{i}z}{1-\mathrm{i}z}=$
$\displaystyle\sum\limits_{n=0}^\infty
\mathrm{i}^nz^n+\mathrm{i}^{n+1}z^{n+1}$, $|z| < 1$
\\\\
\textit{(iv)}
$\dfrac{1}{1-z+z^2}=$
$\dfrac{z+1}{z^3+1}=$
$\dfrac{1}{1+z^3}+\dfrac{z}{1+z^3}=$
$\displaystyle\sum\limits_{n=0}^\infty 
(-1)^n\left(z^{4n}+z^{4n+1}\right)$, $|z| < 1$
\\\\
\textit{(v)}
$\dfrac{1}{(z+1)(z+2)}=$
$\dfrac{1}{z+1} - \dfrac{1}{z+2}=$
$\displaystyle\sum\limits_{n=0}^\infty 
\left(\dfrac{1}{-2^{n+1}}-\dfrac{1}{-1^{n+1}}\right)z^n$, $|z| < 1$
\\\\
\textit{(vi)}
\\\\
\item[\textbf{6.4}] 
\textit{(i)}
From the Ratio Test, we get: \\\\
$\left|\dfrac{-1^{n+1}z^{n+1}}{(n+1)^3}\cdot\dfrac{n^3}{(-1)^nz^n}\right|=$
$\left|\dfrac{zn^3}{(n+1)^3}\right|\rightarrow|z|$ as $n\rightarrow\infty$.\\\\
So we get $|z|<1$, so $R=1$.
\\\\
\textit{(ii)}
From the Ratio Test, we get: \\\\
$\left|\dfrac{z^{5(n+1}}{z^{5n}}\right|=$
$\left|\dfrac{z^{5n+5}}{z^{5n}}\right|=$
$\left|z^5\right|$.\\\\
This converges if $|z^5|<1$, and since $|z^5| = |z|^5$, we get $|z| < 1$, so $R=1$.
\\\\
\textit{(iii)}
From the $n$-th root test, we get: \\\\
$\sqrt[n]{\left|\dfrac{z^n}{n^n}\right|}=$
$\left|\dfrac{z}{n}\right|$,
which goes to 0 as $n\rightarrow\infty$, no matter what $z$. So, $R=\infty$
\\\\
\textit{(iv)}
From the Ratio Test, we get \\\\
$\left|\dfrac{(n+1)!z^{n+1}}{n!z^n}\right|=|(n+1)z|$
which goes to 0 if $z=0$ and $\infty$ otherwise. So $R=0$.
\item[\textbf{6.7}] 
We find these by taking the derivative of $(1+z)^{-1}$, which we know the series expansion of.\\\\
$(1+z)^{-2}$ \\\\
$(1+z)^{-1} = 1-z+z^2-z^3...$ \\
$\dfrac{\mathrm{d}}{\mathrm{d}z}(1+z)^{-1} = 0 - 1 + 2z - 3z^2... = -(1+z)^{-2}$ \\
So, $(1+z)^{-2} = \displaystyle\sum\limits_{n=1}^\infty 
(-1)^{n-1}nz^{n-1}$, $|z| < 1$
\\\\\\
$(1+z)^{-3}$ \\\\
$(1+z)^{-2} = - 1 + 2z - 3z^2...$ \\
$\dfrac{\mathrm{d}}{\mathrm{d}z}(1+z)^{-2} = 0 + 2 - 6z... = -2(1+z)^{-3}$ \\
So, $(1+z)^{-3} = \dfrac{1}{2}\displaystyle\sum\limits_{n=2}^\infty 
(-1)^{n-2}n(n-1)z^{n-2}$, $|z| < 1$
\item[\textbf{6.10}] 
\textit{(i)}
From the Ratio Test, we get: \\\\
$\left|\dfrac{(z+1)^{n+1}}{2^{n+1}}\cdot\dfrac{2^{n}}{(z+1)^{n}}\right|=$
$\left|\dfrac{z+1}{2}\right|$ \\\\
This converges absolutely when $\left|\dfrac{z+1}{2}\right|<1$, so we get the set of points for which $|z| < 1$.
\\\\
\textit{(ii)}
From the Ratio Test, we get: \\\\
$\left|\dfrac{(z-1)^{n+1}}{(z+1)^{n+1}}\cdot\dfrac{(z+1)^{n}}{(z-1)^{n}}\right|=$
$\left|\dfrac{z-1}{z+1}\right|$ \\\\
This converges absolutely when $\left|\dfrac{z-1}{z+1}\right|<1$, so we get the set of points Re $z > 0$.
\\\\
\textit{(iii)}
$\displaystyle\sum\limits_{n=1}^\infty \dfrac{1}{n^2}(z^n+z^{-n}) = $
$\displaystyle\sum\limits_{n=1}^\infty \dfrac{1}{n^2}z^n + $
$\displaystyle\sum\limits_{n=1}^\infty \dfrac{1}{n^2}z^{-n} $ \\\\
We can do Ratio Test on each individual sum.\\\\
For the first we get:
$\left|\dfrac{z^{n+1}}{(n+1)^2}\cdot\dfrac{n^2}{z^{n}}\right|
\rightarrow |z|$ as $n\rightarrow\infty$. So $|z| < 1$. \\\\
For the second we get:
$\left|\dfrac{n^2z^n}{(n+1)^{2}z^{n+1}}\right|
\rightarrow \dfrac{1}{|z|}$ as $n\rightarrow\infty$. So $|z| > 1$. \\\\
Clearly, the sum of the individual sums does not absolutely converge anywhere, since $|z|$ cannot be less than 1 and greater than 1. 
\\\\
\textit{(iv)}
From the Ratio Test, we get:\\\\
$\left|\dfrac{z^{n+1}}{1-z^{n+1}}\cdot\dfrac{1-z^{n}}{z^{n}}\right| =
\left|\dfrac{z(1-z)(1+z+z^2...z^{n-1})}{(1-z)(1+z+z^2...z^{n})}\right| =
\left|\dfrac{z+z^2...z^{n}}{1+z+z^2...z^{n}}\right|$ \\\\
This goes to 1 as $n\rightarrow\infty$, so this does not absolutely converge anywhere.
\end{enumerate}