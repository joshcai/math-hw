\begin{enumerate}

\item[\textbf{5.1}] 
\textit{(i)}
$f(z) = \text{Im} z$ \\
\begin{align*}
u(x,y) = 0 ~~~~&~~~~ v(x,y) = y \\
u_x = 0 ~~~~&~~~~ v_x = 0 \\
u_y = 0 ~~~~&~~~~ v_y = 1 
\end{align*}
$u_x \ne v_y$, so $f$ does not satisfy the Cauchy-Riemann equations.
\\\\
\textit{(ii)}
$f(z) = \overline{z}$ \\
\begin{align*}
u(x,y) = x ~~~~&~~~~ v(x,y) = -y \\
u_x = 1 ~~~~&~~~~ v_x = 0 \\
u_y = 0 ~~~~&~~~~ v_y = -1 
\end{align*}
$u_x \ne v_y$, so $f$ does not satisfy the Cauchy-Riemann equations.

\item[\textbf{5.2}] 
\textit{(i)}
$z^3$ \\
Let $z=x+y$i. Then we have, $z^2 = x^2 - y^2 + 2xy$i, and $z^3 = x^3 - 3xy^2 +(3x^2y-y^3)$i.
So we have:
\begin{align*}
u(x,y) = x^3 - 3xy^2 ~~~~&~~~~ v(x,y) = 3x^2y-y^3 \\
u_x = 3x^2-3y^2 ~~~~&~~~~ v_x = -6xy \\
u_y = 6xy ~~~~&~~~~ v_y = 3x^2-3y^2 
\end{align*}
Since $u_x = v_y$ and $u_y = -v_x$, this satisfies the Cauchy-Riemann equations.
\\\\
\textit{(ii)}
$z+z^{-1}$\\
Let $z=x+y$i. Then we have $z+z^{-1} = x+y\mathrm{i}+\dfrac{1}{x+y\mathrm{i}} = x+y\mathrm{i}+\dfrac{x-y\mathrm{i}}{x^2+y^2}$.
\begin{align*}
u(x,y) = x+\dfrac{x}{x^2+y^2} ~~~~&~~~~ v(x,y) = y-\dfrac{y}{x^2+y^2} \\
u_x = 1+\dfrac{y^2-x^2}{(x^2+y^2)^2} ~~~~&~~~~ v_x = \dfrac{2xy}{(x^2+y^2)^2}\\
u_y = \dfrac{-2xy}{(x^2+y^2)^2} ~~~~&~~~~ v_y = 1+\dfrac{y^2-x^2}{(x^2+y^2)^2}
\end{align*}
Since $u_x = v_y$ and $u_y = -v_x$, this satisfies the Cauchy-Riemann equations.
\\\\
\textit{(iii)}
$\dfrac{1}{1-z}$ \\
Let $z=x+y$i. Then we have $\dfrac{1}{1-z} =
\dfrac{1}{1-(x+y\mathrm{i})}=
\dfrac{1}{(1-x)+y\mathrm{i}}=
\dfrac{(1-x)+y\mathrm{i}}{(1-x)^2+y^2}$.
\begin{align*}
u(x,y) = \dfrac{1-x}{(1-x)^2+y^2}
~~~~&~~~~ 
v(x,y) = \dfrac{y}{(1-x)^2+y^2} \\
u_x = \dfrac{(1-x)^2-y^2}{((1-x)^2+y^2)^2} 
~~~~&~~~~
v_x = \dfrac{2(1-x)y}{((1-x)^2+y^2)^2}\\
u_y = -\dfrac{2(1-x)y}{((1-x)^2+y^2)^2} 
~~~~&~~~~
v_y = \dfrac{(1-x)^2-y^2}{((1-x)^2+y^2)^2}
\end{align*}
Since $u_x = v_y$ and $u_y = -v_x$, this satisfies the Cauchy-Riemann equations.
\\\\
\item[\textbf{5.5}] 
\textit{(a)}
At $z=0$, $f(z) = 0$, so $u(x,y) = 0$ and $v(x,y) = 0$. We have the following partial derivatives: \\
\begin{align*}
u_x = 0 ~~~~&~~~~ v_x = 0 \\
u_y = 0 ~~~~&~~~~ v_y = 0 
\end{align*}
Since $u_x = v_y$ and $u_y = -v_x$, this satisfies the Cauchy-Riemann equations. \\\\
Let $z = x+y$i. We can see that $f$ is not differentiable at $z=0$, since if we approach 0 along the ray arg($z) = 0$, we get $f(z) = x$, but if we approach 0 along the ray arg($z) = \pi/2$, we get $f(z) = yi$ (note: $(y\mathrm{i})^5 = y^5$i). 
\\\\
\textit{(b)}
At $z=0$, we have 
\begin{align*}
u_x = 0 ~~~~&~~~~ v_x = 0 \\
u_y = 0 ~~~~&~~~~ v_y = 0 
\end{align*}
Since $u_x = v_y$ and $u_y = -v_x$, this satisfies the Cauchy-Riemann equations at $z = 0$. \\\\
However, it is not differentiable at $z=0$, since if we approach 0 from the real axis or the imaginary axis, $f(z) = 0$, but if we approach 0 from any other ray, the $f(z) > 0$. 
\item[\textbf{5.6}] 
$f(z) = z^3$, so $f(\mathrm{i}) = -\mathrm{i}$ and $f(1) = 1$ \\\\
$\dfrac{f(\mathrm{i}) - f(1)}{\mathrm{i}-1} =
\dfrac{-\mathrm{i}-1}{\mathrm{i}-1} = 
\dfrac{-\mathrm{i}-1}{\mathrm{i}-1}\cdot\dfrac{-\mathrm{i}-1}{-\mathrm{i}-1} = 
\dfrac{2\mathrm{i}}{2} = \mathrm{i}$
We find $f'(z)$ to be $3z^2$. Let $z = x+\mathrm{i}y$. Then $f'(z) = 3x^2-3y^2 + 6xy\mathrm{i}$. \\
Let $f'(z) = \mathrm{i}$. We have 
\begin{align*}
6xy &= 1 \\
3x^2 - 3y^2 &= 0
\end{align*}
Rearranging and factoring, we get
\begin{align}
xy &= 1/6 \\
(x+y)(x-y) &= 0
\end{align}
From (2), we can see that $x=y$ or $x=-y$. But, $x\ne -y$, since $xy$ is positive. Thus, we get the solution $x=\frac{1}{\sqrt{6}}$, $y=\frac{1}{\sqrt{6}}$, and $z=\frac{1}{\sqrt{6}}+\frac{1}{\sqrt{6}}\mathrm{i}$, which is clearly not on the line segment $[1,\mathrm{i}]$.     
\item[\textbf{5.8}] 
Suppose that $f(z) = \dfrac{z}{1+|z|}$ is holomorphic at a point $a\in \mathbb{C}$. Then there exists $r > 0$ such that $f$ is defined and holomorphic in $D(a;r)$. \\\\
We know that if $f$ is holomorphic, then $1/f$ is holomorphic, so $\dfrac{1+|z|}{z}$ is also holomorphic. \\\\
The product of two holomorphic functions is holomorphic, and we know that $g(z) = z$ is holomorphic, so $(g/f)(z) = 1+|z|$ is holomorphic. \\\\
$h(z) = -1$ is also holomorphic, and since we know the sum of two holomorphic functions is holomorphic, we know that $(g/f+h)(z) = |z|$ is holomorphic. \\\\
This is a contradiction, however, since we know that $|z|$ is not differentiable anywhere. Therefore, $\dfrac{z}{1+|z|}$ is not holomorphic anywhere.
\item[\textbf{5.10}] 
If Re $f$ is constant, then $f(z) = d + v(x,y)\mathrm{i}$, for some $d \in \mathbb{R}$. Since $f$ is holomorphic, $f$ satisfies the Cauchy-Riemann equations. \\\\
Since $u(x,y) = d$, we have $u_x = 0$ and $u_y = 0$. That means that $v_x = -u_y = 0$ and $v_y = u_x = 0$. \\\\
We can now see that $f'(z) = u_x + \mathrm{i}v_x = 0$ for all $z \in G$. We know from Proposition 5.12(1) that this forces $f$ to be constant.
\item[\textbf{5.12}] 
Let $f(x,y) = u(x,y) + v(x,y)\mathrm{i}$. Then, we can let: \\
\[\dfrac{\partial f}{\partial x} = u_x + v_x\mathrm{i}\]
and 
\[\dfrac{\partial f}{\partial y} = u_y + v_y\mathrm{i}\]
We can see that 
\begin{align*}
\dfrac{\partial f}{\partial \overline{z}}
&= \dfrac{1}{2}\left(\dfrac{\partial f}{\partial x}+\mathrm{i}\dfrac{\partial f}{\partial y}\right) \\
&= \dfrac{1}{2}\left(u_x + v_x\mathrm{i} + \mathrm{i}(u_y + v_y\mathrm{i})\right) \\
&= \dfrac{1}{2}\left(u_x + v_x\mathrm{i} + u_y\mathrm{i} - v_y\right) \\
&= \dfrac{1}{2}\left(0 + 0\mathrm{i}\right) \\
&= 0
\end{align*}
We also find that
\begin{align*}
\dfrac{\partial f}{\partial z}
&= \dfrac{1}{2}\left(\dfrac{\partial f}{\partial x}-\mathrm{i}\dfrac{\partial f}{\partial y}\right) \\
&= \dfrac{1}{2}\left(u_x + v_x\mathrm{i} - \mathrm{i}(u_y + v_y\mathrm{i})\right) \\
&= \dfrac{1}{2}\left(u_x + v_x\mathrm{i} - u_y\mathrm{i} + v_y\right) \\
&= \dfrac{1}{2}\left(f' + f'\right) \\
&= f'
\end{align*}
\\\\
Prove that a differentiable function $f$ which satisfies $\partial f/\partial \overline{z} = 0$ in $G$ is holomorphic in $G$. \\\\
? I'm not quite sure I understand what is being asked. If $f$ is differentiable in $G$, shouldn't it also be holomorphic in $G$? 
\end{enumerate}