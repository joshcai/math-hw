\documentclass{article}
\usepackage{graphicx}
\usepackage{amsmath}
\usepackage{tikz}
\usepackage[all]{xy}
\usetikzlibrary{positioning,chains,fit,shapes,calc}

\begin{document}

\title{Homework 4}
\author{Josh Cai}

\maketitle
\section*{Section 9.6}


\textbf{2c)} Not a partial ordering. Is reflexive because $(a,a)$ is in the relation for all $a$ in the set. Is antisymmetric because (2,1) and (1,3) do not exist in the relation. Is not transitive because (3,1) and (1,2) exist in the relation but (3,2) does not.

\noindent\textbf{2d)} Not a partial ordering. Is reflexive because $(a,a)$ is in the relation for all $a$ in the set. Is antisymmetric because (2,1), (3,1), (0,2),(3,2),(0,3) do not exist in the relation. Is not transitive because (1,2) and (2,0) exist in the relation but (1,0) does not.
\\\\\noindent\textbf{4c)} Is a poset. Is reflexive because $a=a$. Is antisymmetric because if $a$ is a descendant of $b$ then $b$ cannot be a descendant of $a$. Is transitive because if $a$ is a descendant of $b$ and $b$ is a descendant of $c$, $a$ is a descendant of $c$.
\\\\\noindent\textbf{6c)} Is a poset. Is reflexive because $a=a$. Is antisymmetric because if $a \le b$, and $b \le a$, then $a=b$. Is transitive because if $a\le b$ and $b\le c$, $a\le c$.

\noindent\textbf{6d)} Is not a poset. Is not reflexive because $a=a$ so $a$ would not be related to itself. Is not antisymmetric 5 is related to 4 and 4 is related to 5, but 5 $\neq$ 4. Is not transitive because 5 is related to 4 and 4 is related to 5 but 5 is not related to 5.
\\\\\noindent\textbf{8c)} Is not a poset. Is reflexive because diagonal is all 1's. Is antisymmetric because no 1 is reflected across the diagonal. Is not transitive because (1,3) and (3,4) exist but (1,4) does not exist in the relation.
\\\\\noindent\textbf{10)} Is not a poset. Is reflexive because all loops are there. Is antisymmetric because all edges are directed in only one direction. Is not transitive because (c,d) and (d,b) exist but (c,b) does not exist in the relation.
\\\\\noindent\textbf{14a)} Is comparable, since 15/5 = 3.

\noindent\textbf{14b)} Is not comparable because 9 $\equiv 3 \mod{6}$.

\noindent\textbf{14c)} Is comparable, since 16/8 = 2.

\noindent\textbf{14d)} Is comparable, since 7/7 = 1.
\\\\\noindent\textbf{16a)} \{(1,1), (1,2), (1,3), (1,4), (2,1), (2,2)\}

\noindent\textbf{16b)} \{(3,2),(3,3),(3,4),(4,1),(4,2),(4,3),(4,4)\}

\noindent\textbf{16c)} Attached at bottom.
\\\\\noindent\textbf{18c)}\textit{zero, zoo, zoological, zoology, zoom}
\\\\\noindent\textbf{20)} Attached at bottom.
\\\\\noindent\textbf{22a)} Attached at bottom.

\noindent\textbf{22b)} Attached at bottom.

\noindent\textbf{22c)} Attached at bottom.

\noindent\textbf{22d)} Attached at bottom.
\\\\\noindent\textbf{24)} Attached at bottom.
\\\\\noindent\textbf{26)} $\{(a,b),(a,c),(a,e),(a,d),(b,e),(b,d),(c,d)\}$
\\\\\noindent\textbf{32a)} $\{l,m\}$

\noindent\textbf{32b)} $\{a,b,c\}$

\noindent\textbf{32c)} No, $l$ and $m$ are not comparable.

\noindent\textbf{32d)} No, $a$ and $b$ are not comparable.

\noindent\textbf{32e)} $\{k,l,m\}$

\noindent\textbf{32f)} $k$ is the least upper bound.

\noindent\textbf{32g)} No lower bounds.

\noindent\textbf{32h)} Does not exist.
\\\\\noindent\textbf{34a)} $\{27,48,70,72\}$

\noindent\textbf{34b)} $\{2,9\}$

\noindent\textbf{34c)} No, 60 and 72 are not comparable.

\noindent\textbf{34d)} No, 2 and 9 are not comparable.

\noindent\textbf{34e)} $\{18,36,72\}$

\noindent\textbf{34f)} 18 is the least upper bound.

\noindent\textbf{34g)} $\{2,4,6,12\}$

\noindent\textbf{34h)} 12 is the greatest lower bound.




\end{document}