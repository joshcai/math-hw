\documentclass{article}
\usepackage{graphicx}
\usepackage{amsmath}
\usepackage{tikz}
\usepackage[all]{xy}
\usetikzlibrary{positioning,chains,fit,shapes,calc}

\begin{document}

\title{Homework 8}
\author{Josh Cai}

\maketitle
\section*{Section 10.4}

\textbf{2a)} $f(1) = -6, f(2) = 12, f(3) = -24, f(4) = 48, f(5) = -96$

\noindent\textbf{2b)} $f(1) = 16, f(2) = 55, f(3) = 172, f(4) = 523, f(5) = 1576$

\noindent\textbf{2c)} $f(1) = 1, f(2) = -3, f(3) = 13, f(4) = 141, f(5) = 19597$
\\\\\noindent\textbf{4a)} $f(2) = 0, f(3) = -1, f(4) = -1, f(5) = 0$

\noindent\textbf{4b)} $f(2) = 1, f(3) = 1, f(4) = 1, f(5) = 1$

\noindent\textbf{4c)} $f(2) = 2, f(3) = 5, f(4) = 33, f(5) = 1214$
\\\\\noindent\textbf{6b)} $f(n) = 0$ if $n \equiv 1 \mod{3}$ and $f(n) = 2^{\lfloor (n+1)/3 \rfloor}$ otherwise
\\Basis step: $f(0) = 1, f(1) = 0, f(2) = 2$
\\Recursive step: $f(n) = 2f(n-3) = 2*0 = 0$ if $n \equiv 1 \mod{3}$
\\$f(n) = 2f(n-3) = 2*2^{\lfloor (n-2)/3 \rfloor} = 2^{\lfloor (n-2)/3 \rfloor+1} = 2^{\lfloor (n+1)/3 \rfloor}$ otherwise
\\\\\noindent\textbf{8a)} $f(1) = 2, f(n+1) = f(n)+4$

\noindent\textbf{8b)} $f(1) = 2, f(n+1) = f(n)+2(-1)^{n+1}$

\noindent\textbf{8c)} $f(1) = 2, f(n+1) = f(n)(n+2)/(n)$

\noindent\textbf{8d)} $f(1) = 1, f(n+1) = f(n)+2n+1$
\\\\\noindent\textbf{10)} $f_m(0)=m, f_m(n+1) = f_m(n)+1$
\\\\\noindent\textbf{24a)} $1\in S; n\in S \Rightarrow n-2\in S \wedge n+2 \in S$

\\\\\noindent\textbf{24b)} $1\in S; n\in S \Rightarrow 3n\in S$

\\\\\noindent\textbf{24c)} $x\in S \wedge 1 \in S; a\in S \wedge b\in S \Rightarrow a+b\in S \wedge a-b\in S \wedge ab\in S$
\\\\\noindent\textbf{32a)} $ones(\lambda) = 0; ones(s1) = ones(s)+1; ones(s0)=ones(s)$

\\\\\noindent\textbf{32b)} Basis step: $ones(x\lambda)=ones(x)=ones(x)+0=ones(x)+ones(\lambda)$, so $P(\lambda)$ is true.
\\Recursive step: Assume $P(y)$ true, need to show $ones(xy1)=ones(x)+ones(y1)$ and $ones(xy0)=ones(x)+ones(y0)$. Since $ones(xy1) = ones(xy)+1$, $ones(y1)=ones(y)+1$, and $ones(xy)=ones(x)+ones(y)$ by assumption, $ones(xy1)=ones(xy)+1=ones(x)+ones(y)+1=ones(x)+ones(y1)$. Since $ones(xy0) = ones(xy)$, $ones(y0)=ones(y)$, and $ones(xy)=ones(x)+ones(y)$ by assumption, $ones(xy0)=ones(xy)=ones(x)+ones(y)=ones(x)+ones(y0)$. 
\\\\\noindent\textbf{38)}  $1\in S \wedge 0 \in S; s\in S \Rightarrow 1s1\in S \wedge 0s0\in S$
\\\\\noindent\textbf{40)} $0\in S; s\in S \Rightarrow 0s,s0,s10,s01,1s0,10s,0s1,01s\in S$















\end{document}