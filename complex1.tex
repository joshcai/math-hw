\documentclass{article}
\usepackage{graphicx}
\usepackage{amsmath}

\topmargin=-0.45in
\evensidemargin=0in
\oddsidemargin=0in
\textwidth=6.5in
\textheight=9.0in
\headsep=0.25in


\begin{document}

\title{Complex Analysis}
\author{Josh Cai}

\maketitle  

\section{Chapter 1}
% \begin{enumerate}

\item[\textbf{1.7}] 
\textit{(i)}
Let $z = r\mathrm{e}^{\mathrm{i}\theta}$.
\begin{align*}
    z^3+1 &= 0 \\
    z^3   &= -1 \\
    r^3\mathrm{e}^{3\mathrm{i}\theta} &= -1
\end{align*}
Then, $r^3 = |-1|$, so $ r=1$, and we have:
\begin{gather*}
  \cos 3\theta + \mathrm{i} \sin 3\theta = -1 \\
  \cos 3\theta = -1 \\
  3\theta = (2k+1)\pi, \text{for } k = 0, 1, 2, ...
\end{gather*}
So we have $\theta = \pi/3, \pi, 5\pi/3$, therefore 
$z = \mathrm{e}^{\pi/3\mathrm{i}}, \:
-1, \:
\mathrm{e}^{5\pi/3\mathrm{i}}$.
\\\\
\textit{(ii)}
Let $z = r\mathrm{e}^{\mathrm{i}\theta}$.
\begin{align*}
    z^4+1 &= 0 \\
    z^4   &= -1 \\
    r^4\mathrm{e}^{4\mathrm{i}\theta} &= -1
\end{align*}
Then, $r^4 = |-1|$, so $ r=1$, and we have:
\begin{gather*}
  \cos 4\theta + \mathrm{i} \sin 4\theta = -1 \\
  \cos 4\theta = -1 \\
  4\theta = (2k+1)\pi, \text{for } k = 0, 1, 2, ...
\end{gather*}
So we have $\theta = \pi/4, 3\pi/4, 5\pi/4, 7\pi/4$, 
therefore $z = \mathrm{e}^{\pi/4\mathrm{i}}, \:
\mathrm{e}^{3\pi/4\mathrm{i}}, \:
\mathrm{e}^{5\pi/4\mathrm{i}}, \:
\mathrm{e}^{7\pi/4\mathrm{i}}$.
\\\\
\textit{(iii)}
Let $z = r\mathrm{e}^{\mathrm{i}\theta}$.
\begin{align*}
    z^6+1 &= 0 \\
    z^6   &= -1 \\
    r^6\mathrm{e}^{6\mathrm{i}\theta} &= -1
\end{align*}
Then, $r^6 = |-1|$, so $ r=1$, and we have:
\begin{gather*}
  \cos 6\theta + \mathrm{i} \sin 6\theta = -1 \\
  \cos 6\theta = -1 \\
  6\theta = (2k+1)\pi, \text{for } k = 0, 1, 2, ...
\end{gather*}
So we have 
$\theta = \frac{\pi}{6}, \pi/2, 5\pi/6, 7\pi/6, 3\pi/2, 11\pi/6$, 
therefore 
$z = \mathrm{e}^{\pi/6\mathrm{i}}, \:
\mathrm{e}^{\pi/2\mathrm{i}},\:
\mathrm{e}^{5\pi/6\mathrm{i}},\:
\mathrm{e}^{7\pi/6\mathrm{i}},\:
\mathrm{e}^{3\pi/2\mathrm{i}},\:
\mathrm{e}^{11\pi/6\mathrm{i}}
$.

\item[\textbf{1.8}] 
\textit{(i)}
$1+z+...+z^7 = \dfrac{z^8-1}{z-1}$

The solutions of $\dfrac{z^8-1}{z-1} = 0$ are the 8th roots of unity, except for 1. 

So, $z = \mathrm{e}^{\pi/4\mathrm{i}}, \:
\mathrm{i},\:
\mathrm{e}^{3\pi/4\mathrm{i}},\:
-\mathrm{i},\:
\mathrm{e}^{5\pi/4\mathrm{i}},\:
-1, \:
\mathrm{e}^{7\pi/4\mathrm{i}}
$.
\\\\
\textit{(ii)}
\begin{gather*}
  (1-z)^6 = (1+z)^6 \\
  (1-z)^6 - (1+z)^6 = 0 \\
  ((1-z)^3)^2 - ((1+z)^3)^2 = 0 \\
  ((1-z)^3 + (1+z)^3)((1-z)^3 - (1+z)^3) = 0 
\end{gather*}
\\
$(1-z)^3 = 1 - 3z + 3z^2 - z^3$ and 
$(1+z)^3 = 1 + 3z + 3z^2 + z^3$, so we get:
\begin{gather*}
  (1+3z^2)(6z+2z^3) = 0 \\
  (1+3z^2)(3+z^2)z = 0 \\
\end{gather*}
So, $z=0, \:
\sqrt{3}\mathrm{i},  \:
-\sqrt{3}\mathrm{i}, \:
\dfrac{\mathrm{i}}{\sqrt{3}}, \:
-\dfrac{\mathrm{i}}{\sqrt{3}}
$.
\\\\
\textit{(iii)}
$1-z+z^2 = \dfrac{z^3+1}{z+1}$

The solutions of $\dfrac{z^3+1}{z+1} = 0$ are the solutions of $z^3+1 = 0$ (found in 1.7\textit{(i)}), except for -1. 

So, $z = \mathrm{e}^{\pi/3\mathrm{i}}, \:
\mathrm{e}^{5\pi/3\mathrm{i}}
$.
\\\\
\textit{(iv)}
$1-z^2+z^4-z^6 = \dfrac{1-z^8}{z^2+1}$

The solutions of $\dfrac{1-z^8}{z^2+1} = 0$ are the 8th roots of unity, except for i and -i. 

So, $z = 1, \:
\mathrm{e}^{\pi/4\mathrm{i}}, \:
\mathrm{e}^{3\pi/4\mathrm{i}},\:
\mathrm{e}^{5\pi/4\mathrm{i}},\:
-1, \:
\mathrm{e}^{7\pi/4\mathrm{i}}
$.
\\\\
\item[\textbf{1.11}]
Let $z = x+y\mathrm{i}$ and $w = a+b\mathrm{i}$.

$|z+\mathrm{i}w|^2+|w+\mathrm{i}z|^2$
\begin{align*}
  &=(\sqrt{(x-b)^2 + (a+y)^2})^2 + 
  (\sqrt{(a-y)^2 + (x+b)^2})^2 \\
  &=(x-b)^2 + (a+y)^2 + (a-y)^2 + (x+b)^2 \\  
  &= 2x^2 + 2y^2 + 2a^2 + 2b^2 \\
  &= 2(x^2 + y^2 + a^2 + b^2) \\
  &= 2(|z|^2 + |w|^2)
\end{align*}

\item[\textbf{1.12}]
\textit{(a)} \\
$|1-\overline{z}w|^2 - |z-w|^2$
\begin{align*}
  &= (1-\overline{z}w)(\overline{1-\overline{z}w})
  - (z-w)(\overline{z-w}) \\
  &= (1-\overline{z}w)(1-z\overline{w})
  - (z-w)(\overline{z}-\overline{w}) \\
  &= 1-z\overline{w}-w\overline{z}+|z|^2|w|^2-|z|^2
  +z\overline{w}+w\overline{z}-|w|^2 \\
  &= 1+|z|^2|w|^2-|z|^2-|w|^2 \\
  &= (1-|z|^2)(1-|w|^2)
\end{align*}
\textit{(b)} \\
Since $|z| < 1$ and $|w| < 1$,
\begin{align*}
  (1-|z|^2)(1-|w|^2) > 0& \\
  |1-\overline{z}w|^2 - |z-w|^2 > 0& \:\text{ from \textit{(a)}} \\
  |1-\overline{z}w|^2 > |z-w|^2& \\
  |1-\overline{z}w| > |z-w|& \:\text{ since both positive} \\
  1 > \dfrac{|z-w|}{|1-\overline{z}w|}& \\
  |\dfrac{z-w}{1-\overline{z}w}| < 1&
\end{align*}
\item[\textbf{1.14}]
\end{enumerate}
\begin{enumerate}

\item[\textbf{1.7}] 
\textit{(i)}
Let $z = r\mathrm{e}^{\mathrm{i}\theta}$.
\begin{align*}
    z^3+1 &= 0 \\
    z^3   &= -1 \\
    r^3\mathrm{e}^{3\mathrm{i}\theta} &= -1
\end{align*}
Then, $r^3 = |-1|$, so $ r=1$, and we have:
\begin{gather*}
  \cos 3\theta + \mathrm{i} \sin 3\theta = -1 \\
  \cos 3\theta = -1 \\
  3\theta = (2k+1)\pi, \text{for } k = 0, 1, 2, ...
\end{gather*}
So we have $\theta = \pi/3, \pi, 5\pi/3$, therefore 
$z = \mathrm{e}^{\pi/3\mathrm{i}}, \:
-1, \:
\mathrm{e}^{5\pi/3\mathrm{i}}$.
\\\\
\textit{(ii)}
Let $z = r\mathrm{e}^{\mathrm{i}\theta}$.
\begin{align*}
    z^4+1 &= 0 \\
    z^4   &= -1 \\
    r^4\mathrm{e}^{4\mathrm{i}\theta} &= -1
\end{align*}
Then, $r^4 = |-1|$, so $ r=1$, and we have:
\begin{gather*}
  \cos 4\theta + \mathrm{i} \sin 4\theta = -1 \\
  \cos 4\theta = -1 \\
  4\theta = (2k+1)\pi, \text{for } k = 0, 1, 2, ...
\end{gather*}
So we have $\theta = \pi/4, 3\pi/4, 5\pi/4, 7\pi/4$, 
therefore $z = \mathrm{e}^{\pi/4\mathrm{i}}, \:
\mathrm{e}^{3\pi/4\mathrm{i}}, \:
\mathrm{e}^{5\pi/4\mathrm{i}}, \:
\mathrm{e}^{7\pi/4\mathrm{i}}$.
\\\\
\textit{(iii)}
Let $z = r\mathrm{e}^{\mathrm{i}\theta}$.
\begin{align*}
    z^6+1 &= 0 \\
    z^6   &= -1 \\
    r^6\mathrm{e}^{6\mathrm{i}\theta} &= -1
\end{align*}
Then, $r^6 = |-1|$, so $ r=1$, and we have:
\begin{gather*}
  \cos 6\theta + \mathrm{i} \sin 6\theta = -1 \\
  \cos 6\theta = -1 \\
  6\theta = (2k+1)\pi, \text{for } k = 0, 1, 2, ...
\end{gather*}
So we have 
$\theta = \frac{\pi}{6}, \pi/2, 5\pi/6, 7\pi/6, 3\pi/2, 11\pi/6$, 
therefore 
$z = \mathrm{e}^{\pi/6\mathrm{i}}, \:
\mathrm{e}^{\pi/2\mathrm{i}},\:
\mathrm{e}^{5\pi/6\mathrm{i}},\:
\mathrm{e}^{7\pi/6\mathrm{i}},\:
\mathrm{e}^{3\pi/2\mathrm{i}},\:
\mathrm{e}^{11\pi/6\mathrm{i}}
$.

\item[\textbf{1.8}] 
\textit{(i)}
$1+z+...+z^7 = \dfrac{z^8-1}{z-1}$

The solutions of $\dfrac{z^8-1}{z-1} = 0$ are the 8th roots of unity, except for 1. 

So, $z = \mathrm{e}^{\pi/4\mathrm{i}}, \:
\mathrm{e}^{\pi/2\mathrm{i}},\:
\mathrm{e}^{3\pi/4\mathrm{i}},\:
\mathrm{e}^{\pi\mathrm{i}},\:
\mathrm{e}^{5\pi/4\mathrm{i}},\:
\mathrm{e}^{3\pi/2\mathrm{i}}, \:
\mathrm{e}^{7\pi/4\mathrm{i}}
$.
\\\\
\textit{(ii)}
\\\\
\textit{(iii)}
$1-z+z^2 = \dfrac{z^3+1}{z+1}$

The solutions of $\dfrac{z^3+1}{z+1} = 0$ are the solutions of $z^3+1 = 0$ (found in 1.7\textit{(i)}), except for -1. 

So, $z = \mathrm{e}^{\pi/3\mathrm{i}}, \:
\mathrm{e}^{5\pi/3\mathrm{i}}
$.
\\\\
\textit{(iv)}
\\\\

\end{enumerate}
\section{Chapter 2}

\section{Chapter 3}

\end{document}​ 