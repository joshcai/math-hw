\begin{enumerate}

\item[\textbf{3.2}]
\textit{(i)} Im($z$+i) $<$ 2 \\
Open, closure: Im $(z+\mathrm{i}\le2$

\textit{(ii)} $|z-\mathrm{i}| < |z-1|$ \\
Open, closure: $|z-\mathrm{i}| \le |z-1|$ 

\textit{(iii)} $|z+2\mathrm{i}| \ge 2$ \\
Closed

\textit{(iv)} $|z-1+\mathrm{i}| \ge |z-1-\mathrm{i}|$ \\
Closed

\textit{(v)} Im[$(z+\mathrm{i}/(2\mathrm{i})] < 0$ \\
Open, closure: Im[$(z+\mathrm{i}/(2\mathrm{i})] \le 0$

\textit{(vi)} $1 < $ Re $z \le2$ \\
Neither, closure: $1 \le $ Re $z \le2$ 

\textit{(vii)} Re $z\ne0$ \\
Open, closure: $\mathbb{C}$

\textit{(viii)} $|z-1| < 1 \text{ and } |z| = |z-2|$ \\
Neither, closure: $|z-1| \le 1 \text{ and } |z| = |z-2|$
\\\\
\item[\textbf{3.4}]
\textit{(i)} \\
Suppose $S$ is open and is not $\varnothing$. Then for any $z \in S$, there exists $r>0$ such that D$(z;r)\subseteq S$. But, D$(z;r)$ is not finite, so $S$ is not finite, which is a contradiction. Therefore, $S$ is not open if it is finite and non-empty. 
\\\\
\textit{(ii)} \\
First, we show that $S$ has no limit points. Suppose $S$ has a limit point $l$. Since $S$ is finite, let $k$ denote the closest point to $l$ (or one of the closest if there are multiple equidistant). Take $r = |k-l|/2$. Then D$'(l;r) \cap S = \varnothing$, which means that $l$ is not a limit point. Therefore, $S$ has no limit points. Since $S$ contains all of its limit points (since there are none), $S$ is closed.
\\\\
\item[\textbf{3.5}]
\textit{(a)} \\
We first show that $G_1 \cup G_2$ is a region $\implies G_1 \cap G_2 \ne \varnothing$:

Since $G_1 \cup G_2$ is a region, then $G_1 \cup G_2$ is polygonally connected. We can take $a \in G_1$ and $b \in G_2$. There must be a path between $a$ and $b$, which means there is at least one element common to both $G_1$ and $G_2$. Thus, $G_1 \cap G_2 \ne \varnothing$

Now we show that  $G_1 \cap G_2 \ne \varnothing \implies G_1 \cup G_2$ is a region:

We see easily that $G_1 \cup G_2$ is open, since $G_1$ and $G_2$ are open. We also see that it is non-empty, since $G_1$ is non-empty (by definition of being a region). We also see that $G_1 \cup G_2$ is polygonally connected, since to get from any $a \in G_1$ to any $b \in G_2$, we can just go through any $c \in G_1 \cap G_2$, since it is non-empty. (Since $c$ is in $G_1 \cap G_2$, it is in $G_1$, so there is a polygonal route from $a$ to $c$, since it is given that $G_1$ is a region. Similarly there is a route from $c$ to $b$.) Since $G_1 \cup G_2$ is a non-empty, open set that is polygonally connected, it is a region. 

\textit{(b)} \\
$G_1 \cup G_2$ is a region given that $G_1 \cap G_2 \ne \varnothing$ and the previous result in \textit{(a)}. Then, $(G_1 \cup G_2) \cap G_3 \ne \varnothing$, since $G_2 \cap G_3 \ne \varnothing$, which means $G_1 \cup G_2 \cup G_3$ is a region, and so on until you get to $N$. 
\\\\
\item[\textbf{3.9}]
\textit{(a)} \\
(i) $\dfrac{1}{n}\mathrm{i}^n$ \\
$|z_n|=\dfrac{1}{n}|\mathrm{i}|^n$ \\
Since $1/n\rightarrow0, z_n\rightarrow0$ \\
\\
(ii) $(1+\mathrm{i})^{-n}$ \\
$|z_n|=|1+\mathrm{i}|^{-n}\rightarrow0$\\
So $z_n\rightarrow0$
\\\\
(iii) $\dfrac{n^2+\mathrm{i}n}{n^2+\mathrm{i}}$ \\
$\{\text{Re }z_n\} = \dfrac{n^4+n}{n^4+1}\rightarrow 1$\\
$\{\text{Im }z_n\} = \dfrac{n^3-n^2}{n^4+1}\rightarrow 0$\\
So $z_n\rightarrow1$
\\\\
\textit{(b)} \\
(i) $\mathrm{i}^n$ \\
$z_n = \mathrm{i}, -1, -\mathrm{i}, 1, \mathrm{i}, ...$ \\
Cycles, so has no limit \\
\\\\
(ii) $(1+\mathrm{i})^{n}$ \\
$|z_n|=|1+\mathrm{i}|^{n}\rightarrow\infty$\\
Size gets bigger and bigger, so has no limit\\
\\\\
(iii) $(-1)^n\dfrac{n}{n+\mathrm{i}}$ \\
$\{\text{Re }z_n\} = (-1)^n\dfrac{n^2}{n+1}\rightarrow\infty$\\
Real part does not converge, so whole sequence cannot converge
\\\\
\item[\textbf{3.11}]
\textit{(a)} \\
(i) $z+|z|^3$ \\
Let $z = u+v\mathrm{i}$. \\
$f(z) = u+iv + |u+iv|^3$ \\
Re $f(z) = u+|u+iv|^3 \rightarrow0$ \\
Im $f(z) = v \rightarrow0$ \\
Since both the real and imaginary parts converge to 0, $f(z)$ converges to 0.
\\\\
(ii) $\dfrac{|z|^2}{z}$ \\
Let $z = u+v\mathrm{i}$. \\
$f(z) = \dfrac{u^2+v^2}{u+v\mathrm{i}} = u-v$i \\
Re $f(z) = u \rightarrow0$ \\
Im $f(z) = -v \rightarrow0$ \\
Since both the real and imaginary parts converge to 0, $f(z)$ converges to 0.
\\\\
(iii) $\dfrac{(\text{Re }z)(\text{Im }z)}{|z|}$ \\\\\\\\\\\\
\textit{(b)} \\
(i) $\dfrac{\overline{z}}{z}$ \\
$z_n = \mathrm{i}, -1, -\mathrm{i}, 1, \mathrm{i}, ...$ \\
$f(z) = -1$ when on imaginary axis.
$f(z) = 1$ when on real axis.
\\\\
(ii) $\dfrac{\overline{z}}{|z|}$ \\
$f(z) = -$i when on imaginary axis but above real axis.
$f(z) = \:$i when on imaginary axis but below real axis.
\\\\
(iii) $\dfrac{\text{Im }z}{\text{Re }z}$ \\
$f(z) = 0$ when $z$ is real and $f(z) = 1$ on line $y=x$.
\\\\
\item[\textbf{3.13}]
\textit{(i)} \\
Let $z = a+b\mathrm{i}$. Then $f(z) = 
\dfrac{a+b\mathrm{i}}{1+\sqrt{a^2+b^2}}$.\\
Re $f(z) = \dfrac{a}{1+\sqrt{a^2+b^2}}$, which is continuous.\\
Im $f(z) = \dfrac{b}{1+\sqrt{a^2+b^2}}$, which is also continuous. \\
Since both the real and imaginary parts of $f$ are continuous, $f$ is also continuous.

\textit{(ii)} \\
Let $z = r\mathrm{e}^{\mathrm{i}\theta}$. Then $f(z) = \dfrac{z}{1+|z|} = \dfrac{r\mathrm{e}^{\mathrm{i}\theta}}{1+r}$

Suppose $f(z_1) = f(z_2)$. Then,
\begin{align*}
\dfrac{r_1\mathrm{e}^{\mathrm{i}\theta_1}}{1+r_1} &= 
\dfrac{r_2\mathrm{e}^{\mathrm{i}\theta_2}}{1+r_2} \\
\dfrac{\mathrm{e}^{\mathrm{i}\theta_1}}{\mathrm{e}^{\mathrm{i}\theta_2}} &=
\dfrac{r_2+r_1r_2}{r_1+r_1r_2} \\
\mathrm{e}^{\mathrm{i}(\theta_1 - \theta_2)} &=
\dfrac{r_2+r_1r_2}{r_1+r_1r_2} 
\end{align*}
We see that the right hand side has no complex part, since $r_1$ and $r_2$ are real numbers. So we know that
\[\sin(\theta_1 - \theta_2) = 0\]
which means that
\[\theta_1 - \theta_2 = 0 \text{ or } \theta_1 - \theta_2 = \pi\]
If $\theta_1 - \theta_2 = \pi$, then we get 
\[\mathrm{e}^{\mathrm{i}\pi} = -1 = \dfrac{r_2+r_1r_2}{r_1+r_1r_2} \]
but this is impossible, because $r_1$ and $r_2$ are non-negative.

So $\theta_1 - \theta_2 = 0$, which means $\theta_1 = \theta_2$ and
\begin{align*}
\mathrm{e}^{0} = 1 &= \dfrac{r_2+r_1r_2}{r_1+r_1r_2} \\
r_2+r_1r_2 &= r_1+r_1r_2 \\
r_2 &= r_1
\end{align*}
Since $\theta_1 - \theta_2 = 0$ and $r_2 = r_1$, we know $z_1$ = $z_2$.\\
\textit{(iii)} \\
We want to show that for all $y \in \mathrm{D}(0,1)$ there exists $x$ such that $f(x) = y.$

Let $y = r\mathrm{e}^{\mathrm{i}\theta}$, $r < 1$.
Then $r\mathrm{e}^{\mathrm{i}\theta} = 
\dfrac{r'\mathrm{e}^{\mathrm{i}\theta'}}{1+r'}$.
We set $r = \dfrac{r'}{1+r'}$ and solve for $r'$.
\begin{align*}
  r = \dfrac{r'}{1+r'} \\
  r+rr' = r' \\
  r = r'- rr' \\
  r = r'(1-r) \\
  \dfrac{r}{1-r} = r'
\end{align*}
So for every $y$ represented as $r\mathrm{e}^{\mathrm{i}\theta}$ is simply the image of $\dfrac{r}{1-r}\mathrm{e}^{\mathrm{i}\theta}$.
\\\\
\item[\textbf{3.16}]
\textit{(i)} \\
We want to prove that $\{z_n(c)\}$ tends to infinty by showing that the sequence is strictly increasing.
\begin{align*}
|z_{k+1}| &= |z_{k}^2+c| \\
&\ge ||z_k|^2 - |c|| \\
&= |z_k|^2 - |c| \text{ since } |c| \le 2
\end{align*}
We now want to show that $|z_k|^2 - |c|$ is strictly greater than $|z_k|$.

We can easily see that $|z_k|(|z_k-1|) > |c|$ is true, since the left hand side must be greater than 2, and the right hand side is less than or equal to 2. 

From that we can find:
\begin{align*}
|z_k|(|z_k-1|) > |c| \\
|z_k|^2 - |z_k| > |c| \\
|z_k|^2 - |c| > |z_k|
\end{align*}
We can repeat this process and see that each term gets larger, so the sequence approaches infinity. (Actually, this doesn't sound quite right...)

\textit{(ii)} \\\\\\\\\\\\\\\\\\\\\\\\

\textit{(iii)} \\
To prove that $M$ is closed, we prove that $\mathbb{C} - M$ is open. 

$\mathbb{C} - M$ is the set of $c \in \mathbb{C}$ such that $|c| > 2$ or the sequence $\{z_n(c)\}$ tends to infinity.

The set of $c \in \mathbb{C}$ such that the sequence $\{z_n(c)\}$ tends to infinity is the union $\cup_{k=1}^{\infty} U_k$. Since each of these is open from part \textit{(ii)}, the whole union is open. 

The set of $c \in \mathbb{C}$ such that $|c| > 2$ is simply the set of points outside the circle centered at the origin with radius 2. However, this set is already contained within the previous set, since it is equivalent to $U_1$. 

Since the set of $c \in \mathbb{C}$ such that the sequence $\{z_n(c)\}$ tends to infinity is open, $\mathbb{C} - M$ is open, and $M$ is closed. 
\end{enumerate}